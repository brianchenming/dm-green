\section{Introduction}
\label{intro}

Increasing IT demand has seen 6X growth of server and 69X growth of
storage during the last decade \cite{ibm_green_beyond}. Data centers
energy use is doubling every 5 years, whereas the global electricity
prices are increasing 10-25\% per year. This makes energy efficiency
of computer systems a big concern recently in academic and industrial
communities. Studies show that storage systems are responsible not
only for the system performance bottleneck but also for up to 40\% of
the power consumption\cite{storage_40}. This situation is largely
caused by the inherent drawback of traditional hard disks (HHD), which
consist of physical moving parts. They consume a lot of power to keep
disk platters spin at a high speed and they are slow as disk heads
have to move to the right position before serving random I/O access.

Thanks to the development of solid state drive (SSD) technology, many
of the problems related to performance and energy efficiency are
alleviated. However, despite of having the above advantages, SSD are
not widely adopted in the server storage stack due to their high
capital cost per byte. To overcome this limit, hybrid disks are
proposed. Hybrid disks achieve better trade-off between performance
and capacity by organizing different disks in the manner conformant to
the storage hierarchy. 

The most common type of hybrid disks \cite{Bisson07ahybrid} tries to
reap the benefits of SSDs by having a small flash memory on each hard
disk and using this flash memory as non-volatile cache for the hard
disk. A host can use this non-volatile cache to achieve faster random
reads since it is fast and takes constant time to access any block in
the SSD cache. Though this type of hybrid disks improves performance
over traditional disks, the improvement was modest due to the
relatively small size of non-volatile cache on each disk. It is also
difficult to balance cache load among disks because the SSD caches are
separated. For example, if a workload accesses one disk much more
frequently than others, then this would result in frequent cache
misses on that particular disk, whereas the caches on other disks are
under-utilized. 

To enable servers to benefit more from the power efficiency and high
performance of solid state media, we used a single but relatively
large SSD as cache (it has similar effect as cache but is not exactly
cache) for all other disks (referred to as secondary disks). This
approach would achieve a good cache hit ratio because of the large
size of the cache. Thanks to data locality, it would also make the SSD
capable of serving majority of the disk accesses as we can see in
\cite{Debnath_SkimpyStash, Debnath_Bloomflash,
flashcache_experiments}. Therefore, other disks are idle for longer
periods of time and thus can be powered down to save energy. 

The difference between the above two types of hybrid disks seems to be
not very clear if we consider the virtualization of disk such as Linux
Logical Volume Management (LVM). By grouping all hard disks to form a
single virtual disk, the cache on all these hard disks can be gathered
together which becomes equivalent to the large SSD we are using. This
workaround alleviates the load balance problem. However, it is less
helpful in saving energy because the virtualization of multi-disk is
not aware of energy efficiency. If there are non-adjacent blocks which
are often simultaneously accessed, we can perform data grouping and
store them onto one physical disk so that only this disk has to spin
up when they are accessed. In this case, the optimization cannot be
achieved if the LVM virtualization approach is employed. 

To exploit spatial and temporal data locality to the largest extent,
we collected workload-specific traces and are trying to find hot data
and block groups of working sets. For hot data, we simply find blocks
which are most frequently accessed. We realize that hot data changed
over time, however this offline study is still helpful because some
data is inherent hot, e.g., filesystem metadata and latest data.
Moreover, because our analysis is workload specific it is likely the
same I/O pattern will repeat in same or similar workload. We also try
to keep hot data in cache online using the LRU heuristic algorithm. To
identify block groups, we will use methods in \cite{Wildani_grouping}.
Blocks fall into same group but spread multiple disk will be mapped to
same physical disk so that less disks are required to power up.
Additionally, data grouping has the benefit of isolating faults
\cite{Sivathanu_dgraid, Wildani_grouping}. 

We use the Linux Device Mapper framework and implement our hybrid
model as a Device Mapper target. This provides modularity and
transparency as it enables us to create virtual disks without exposing
the details of underlying physical disks. By leveraging the strengths
of different disks, our virtual disk can better trade-off among power
consumption, I/O performance, and storage capacity.

The rest of the paper is organized as follows. Section
\ref{sec:design} describes our design and implementation. Section
\ref{sec:trace} presents some block trace analysis.  The energy
efficiency and I/O performance of our system is evaluated in Section
\ref{sec:eval}. Related work is discussed in Section
\ref{sec:related}. Conclusions are drawn in Section \ref{sec:conc}.

%%%%%%%%%%%%%%%%%%%%%%%%%%%%%%%%%%%%%%%%%%%%%%%%%%%%%%%%%%%%%%%%%%%%%%%%%%%%%%
%% For Emacs:
% Local variables:
% fill-column: 70
% End:
%%%%%%%%%%%%%%%%%%%%%%%%%%%%%%%%%%%%%%%%%%%%%%%%%%%%%%%%%%%%%%%%%%%%%%%%%%%%%%
%% For Vim:
% vim:textwidth=70
%%%%%%%%%%%%%%%%%%%%%%%%%%%%%%%%%%%%%%%%%%%%%%%%%%%%%%%%%%%%%%%%%%%%%%%%%%%%%%
% LocalWords:  
